\documentclass{simurgh-doc}
\pagestyle{docheadings}
\setmainfont[Ligatures=TeX,Numbers=OldStyle]{Iwona}
\newminted{tex}{frame=leftline,framerule=5pt,rulecolor=\color{gray}}
\newminted{lua}{frame=leftline,framerule=4pt,rulecolor=\color{lightgray}}
\usemintedstyle{emacs}
\makeindex
\renewcommand*{\baselinestretch}{1.5}
\title{\textcolor{red!65!black}{\Huge The \simurgh{} \textit{package}}\\
\LARGE Typeset Parsi in Lua\LaTeX\\[5pt]
\large Documentation for \textcolor{red!65!black}{version 0.01b}}
\author{Vafa Khalighi\\
\url{persian-tex@tug.org}}
\begin{document}
\setcounter{page}{0}
\thispagestyle{empty}
\includegraphics[width=.3\textwidth]{simurgh-logo}\qquad
\parindent 0pt
\vbox{\Huge
\textcolor{red!65!black}{The \simurgh{} \textit{package}}

\Large
Documentation for \textcolor{red!65!black}{version 0.01b}}





\vskip 0pt plus 3fill
\textit{\large to my Master, \textbf{Ferdowsi The Great}}


\vskip 0pt plus 3fill
{\color{red!85!black}\fbox{\begin{minipage}{\textwidth}
This document will eventually become the documentation of the \simurgh{} \textit{package}.
At the moment, it simply reports the behaviour of the \textit{package}
matching the  release version in the title page.

\medskip
Features may come and go. The current version of \simurgh{} \textit{package} is not
meant for production and users can not depend on stability, nor on
functionality staying the same.

\medskip
Nothing is considered stable just yet. This documentation therefore
simply reflects the current state of the \textit{package}. \textbf{\textit{
Absolutely nothing}} on the following pages is set in stone. When
the need arises, anything can (and will) be changed.

\medskip
\textbf{If you are not willing to deal with this situation, you should
wait for the stable version. Currently there is no set date for the stable version. Full stabilization will not happen soon, the
TODO list is still large.}
\end{minipage}}}


\vskip 0pt plus 3fill


Copyright 2013 by Vafa Khalighi

\medskip
Permission is granted to copy, distribute and/or modify all files of this package under the terms of the \textsc{gnu} Public License, Version 2 or any later version published by the Free Software Foundation. A copy of the license is included in the section entitled \textsc{gnu} Public License.


\clearpage








\maketitle


\tableofcontents

\clearpage


\section{Introduction}

\subsection{Overview of the package}

\subsection{How to Read This Documentation}

This documentation explains the commands of the \simurgh{} \textit{package} and its usage. The ``public'' commands and environments provided by the \simurgh{} \textit{package} are described throughout the text. In each such description, the described command, environment or option is printed in red. Text shown in green is optional and can be left out.

In the following documentation, the installation is explained first, followed by an overview of the basic concepts used. Then, we explain the usage of the \textit{package}.

\subsection{Contributing}

\subsection{Getting Help}

When you need help with the \textit{package}, please do the following:

\begin{enumerate}
  \item Read this documentation, at least the part that has to do with your problem.
  \item Consider rereading the documentation, especially the part that has to do with your problem.
  \item If that does not solve the problem, \emph{as a last resort} you can try to email me. I do not mind getting emails, I simply get way too many of them. Because of this, I can not guarantee that your emails will be answered timely or even at all. 
  \item Before you file a bug report, especially a bug report concerning the installation, make sure that this is really a bug. In particular, have a look at the |.log| file that results when you run |lualatex| on your files. This |.log| file should show that all the right files are loaded from the right directories. Nearly all installation problems can be resolved by looking at the |.log| file.
\end{enumerate}

\section{Installation}

This \textit{package} is distributed under the \textsc{gpl} license, see \autoref{subsection:gpl}. 

Typically, the \textit{package} will already be installed on your system. Naturally, in this case you do not need to worry about the installation process at all and you can skip the rest of this section. 

\subsection{\textit{package} and Driver Versions}

This documentation is part of version 0.01b of the \simurgh{} \textit{package}. To use it, you just need a reasonably up-to-date \LaTeX-system, but the requirements are rather low. There are no special dependencies.

\subsection{Installing Prebundled Packages}

I do not create or manage prebundled packages of \simurgh, but perhaps other people might be willing to do so. If you have a problem with installing such a bundle, you might wish to have a look at the page of whoever created the package.


\subsection{Installation in a texmf Tree}

For a permanent installation, you place the files of the \simurgh{} \textit{package} in an appropriate |texmf| tree. 

When you ask |lualatex| to use a certain class or package, it usually looks for the necessary files in so-called |texmf| trees. These trees are simply huge directories that contain these files. By default, |lualatex| looks for files in three different |texmf| trees:
\begin{itemize}
  \item The root |texmf| tree, which is usually located at |/usr/share/texmf/| or |c:\texmf\| or somewhere similar.
  \item  The local  |texmf| tree, which is usually located at |/usr/local/share/texmf/| or |c:\localtexmf\| or somewhere similar.
  \item  Your personal  |texmf| tree, which is usually located in your home directory at |~/texmf/| or |~/Library/texmf/|.   
\end{itemize}

You should install the \textit{package} either in the local tree or in your personal tree, depending on whether you have write access to the local tree. Installation in the root tree can cause problems, since an update of the whole \TeX\ installation will replace this whole tree.

Inside whatever texmf-tree that you have chosen, create the sub-directory |texmf/tex/lualatex/simurgh| and put all the |.cls| and |.sty| files from the \textit{package} in this directory.

Finally, you may need to run the program |texhash| to rebuild \TeX's cache. In Mik\TeX, there is a menu option for this.

\subsection{Updating the Installation}

To update your installation from a previous version, all you need to do is to replace everything in the directory |texmf/tex/lualatex/simurgh| with the files of the new version. The easiest way to do this is to first delete the old version and then proceed as described above.

Sometimes, there are changes in the syntax of certain command from version to version. If things no longer work that used to work, you may wish to have a look at the documentation and at the change log. 

\section{Basic Concepts}

\subsection{The \simurgh{} \textit{package} logo}
The logo of the \simurgh{} \textit{package} is simurgh shown on the titlepage and footer of this documentation.




\subsection{Package Loading Process}
\begin{package}{simurgh}
You load the package in the ordinary way.
\end{package}





\section{Usage}


\subsection{Basic Usage}
\example
\begin{texcode}
\documentclass{article}
\usepackage{simurgh}
\setmainparsifont{XB Kayhan}
\begin{document}
  
\end{document}
\end{texcode}


\subsection{Options of the \textit{Package}}

\begin{packageoption}{debug footnote direction=\meta{value}}
\meta{value} can be any of the following values:
\begin{description}
\item[\defaultkeyval\opt{off}] Does not write the direction of the first footnote on each page, in the terminal and in the log file.
\item[\opt{on}] Writes the direction of the first footnote on each page, in the terminal and in the log file.
\end{description}
\end{packageoption}

\begin{packageoption}{footnote direction=\meta{value}}
\meta{value} can be any of the following values:
\begin{description}
\item[\opt{left to right}] The direction of |\footnote| is left to right.
\item[\defaultkeyval\opt{right to left}] The direction of |\footnote| is right to left.
\end{description}
\end{packageoption}

\begin{packageoption}{footnote rule=\meta{value}}
\meta{value} can be any of the following values:
\begin{description}
\item[\defaultkeyval\opt{automatic}] Left or right aligned footnote rule based on the direction of the first footnote following the rule (i.e. put in the current page).
\item[\opt{left}] Left aligned footnote rule.
\item[\opt{right}] Right aligned footnote rule.
\item[\opt{split}] Puts a full-width rule above the split-off part of a split footnote.
\item[\opt{textwidth}] Full-width footnote rule.
\end{description}
\end{packageoption}


\begin{packageoption}{main direction=\meta{value}}
\meta{value} can be any of the following values:
\begin{description}
\item[\opt{left to right}] The main direction of the document is left to right. This value has no practical applications in Parsi typesetting (it  only exists for debugging purposes).
\item[\defaultkeyval\opt{right to left}] The main direction of the document is right to left. 
\end{description}
\end{packageoption}


\begin{packageoption}{math digits=\meta{value}}
\meta{value} can be any of the following values:
\begin{description}
\item[\defaultkeyval\opt{automatic}] If you use |\setmathparsidigitfont| at least once, then in Parsi mode you get Parsi digits in math mode and in Latin mode you get ``Latin Modern'' fonts \footnote{Of course if you have not loaded any math font packages.} and Latin digits in math mode, otherwise you get \LaTeX{} default digits in math mode everywhere.
\item[\opt{default}] \LaTeX{} default font and digits in math mode everywhere.
\item[\opt{parsi}] Parsi digits in math mode everywhere.
\end{description}
\end{packageoption}

\begin{packageoption}{math direction=\meta{value}}
\meta{value} can be any of the following values:
\begin{description}
\item[\defaultkeyval\opt{left to right}] The direction of math is left to right.
\item[\opt{right to left}] The direction of math is right to left.
\end{description}
\end{packageoption}


\begin{packageoption}{unicode bidi=\meta{value}}
\meta{value} can be any of the following values:
\begin{description}
\item[\defaultkeyval\opt{global}] Activates unicode bidi and does not obey groupings\footnote{There is no concept of grouping in unicode bidi}.
\item[\opt{local}] Activates unicode bidi and obeys groupings.
\item[\opt{off}] Turns unicode bidi off.
\end{description}
\end{packageoption}


\subsection{Font Commands}
\subsubsection{Text}
\paragraph{Basic Font Commands}
\begin{command}{\setmainparsifont\oarg{options}\marg{font name}}
Chooses the main Parsi font. If you do not use this command at least once, then you get an error asking you to choose the main Parsi font.
\end{command}




\begin{command}{\setmainlatinfont\oarg{options}\marg{font name}}
Chooses the main Latin font. If you do not use this command at least once, then ``Latin Modern'' fonts will be used instead.
\end{command}

\paragraph{Loading New Parsi and Latin Fonts}
\begin{command}{\loadnewparsifont\myprintocmmand{fontcs}\oarg{options}\marg{font name}}
Loads new Parsi font.
\example
\begin{texcode}
\documentclass{article}
\usepackage{simurgh}
\setmainparsifont{XB Kayhan}
\loadnewparsifont\nastalighfont[Scale=1]{IranNastaliq}
\begin{document}
  \nastalighfont
  
\end{document}
\end{texcode}
\end{command}


\begin{command}{\loadnewlatinfont\myprintocmmand{fontcs}\oarg{options}\marg{font name}}
Loads new Latin font.
\example
\begin{texcode}
\documentclass{article}
\usepackage{simurgh}
\setmainparsifont{XB Kayhan}
\loadnewlatinfont\timesfont[Scale=1]{Times New Roman}
\begin{document}

  \begin{LTRpar}
    \timesfont
    
  \end{LTRpar}
  
\end{document}  
\end{document}
\end{texcode}
\end{command}

\paragraph{Parsi Sans Font Commands}
\begin{command}{\setparsisansfont\oarg{options}\marg{font name}}

\end{command}


\begin{command}{\parsisffamily}


\end{command}

\begin{command}{\textparsisf\marg{text}}

\end{command}

\paragraph{Latin Sans Font Commands}
\begin{command}{\setlatinsansfont\oarg{options}\marg{font name}}

\end{command}

\begin{command}{\sffamily}

\end{command}


\begin{command}{\textsf\marg{text}}

\end{command}


\paragraph{Parsi Mono Font Commands}
\begin{command}{\setparsimonofont\oarg{options}\marg{font name}}

\end{command}

\begin{command}{\parsittfamily}

\end{command}


\begin{command}{\textparsitt\marg{text}}


\end{command}

\paragraph{Latin Mono Font Commands}
\begin{command}{\setlatinmonofont\oarg{options}\marg{font name}}

\end{command}


\begin{command}{\ttfamily}

\end{command}

\begin{command}{\texttt\marg{text}}

\end{command}


\paragraph{Parsi Iranic Font Commands}
\begin{command}{\setiranicfont\oarg{options}\marg{font name}}

\end{command}

\begin{command}{\iranicfamily}

\end{command}

\begin{command}{\textiranic\marg{text}}

\end{command}


\paragraph{Parsi Navar Font Commnands}
\begin{command}{\setnavarfont\oarg{options}\marg{options}}

\end{command}


\begin{command}{\navarfamily}


\end{command}

\begin{command}{\textnavar\marg{text}}

\end{command}



\paragraph{Parsi Pook Font Commands}
\begin{command}{\setpookfont\oarg{options}\marg{font name}}

\end{command}

\begin{command}{\pookfamily}

\end{command}

\begin{command}{\textpook\marg{text}}

\end{command}

\paragraph{Parsi Sayeh Font Commands}
\begin{command}{\setsayehfont\oarg{options}\marg{font name}}

\end{command}

\begin{command}{\sayehfamily}

\end{command}

\begin{command}{\textsayeh\marg{text}}

\end{command}




\subsubsection{Math}
\begin{command}{\setmathparsidigitfont\oarg{options}\marg{font name}}
Chooses the Parsi font for digits in math mode. If you use |\setmathparsidigitfont| at least once, then in Parsi mode you get Parsi digits in math mode and in Latin mode you get ``Latin Modern'' font and Latin digits in math mode, otherwise you get \LaTeX{} default digits in math mode everywhere.
\end{command}

\begin{command}{\setmathsfparsidigitfont\oarg{options}\marg{font name}}
Chooses the Parsi font for digits in math mode inside |\mathsf| command.  If you use |\setmathsfparsidigitfont| at least once, then in Parsi mode you get Parsi digits in math mode inside |\mathsf| and in Latin mode you get ``Latin Modern'' font and Latin digits in math mode inside |\mathsf|, otherwise you get ``Latin Modern'' fonts and Latin digits in math mode inside |\mathsf| everywhere.
\end{command}

\begin{command}{\setmathttparsidigitfont\oarg{options}\marg{font name}}
Chooses the Parsi font for digits in math mode inside |\mathtt| command.  If you use |\setmathttparsidigitfont| at least once, then in Parsi mode you get Parsi digits in math mode inside |\mathtt| and in Latin mode you get ``Latin Modern'' font and Latin digits in math mode inside |\mathtt|, otherwise you get ``Latin Modern'' fonts and Latin digits in math mode inside |\mathtt| everywhere.
\end{command}


\subsection{Paragraph Switching Commands}
\begin{command}{\setLTRpar}
Makes the direction of the paragraph left to right. The command does not change the current font.
\end{command}

\begin{command}{\setRTLpar}
Makes the direction of the paragraph right to left. The command does not change the current font.
\end{command}

\subsection{Paragraph Switching Environments}
\begin{environment}{{LTRpar}}
Makes the direction of the paragraph left to right. The environment does not change the current font.
\end{environment}

\begin{environment}{{RTLpar}}
Makes the direction of the paragraph right to left. The environment does not change the current font.
\end{environment}

\begin{environment}{{latinpar}}
Makes the direction of the paragraph left to right. The environment changes the current font to Latin font.
\end{environment}


\begin{environment}{{parsipar}}
Makes the direction of the paragraph right to left. The environment changes the current font to Parsi font.
\end{environment}

\subsection{Typesetting Short left to right and right to left Texts}
\label{subsection:tslrrlt}
\begin{command}{\textLRE\marg{text}}
Typesets short left to right texts in right to left paragraphs. The command does not change the current font.
\end{command}

\begin{command}{\textRLE\marg{text}}
Typesets short right to left texts in left to right paragraphs. The command does not change the current font.
\end{command}

\begin{command}{\textLatin\marg{text}}
Typesets short left to right texts in right to left paragraphs. The command changes the curent font to Latin font.
\end{command}

\begin{command}{\textParsi\marg{text}}
Typesets short right to left texts in left to right paragraphs. The command changes the current font to Parsi font.
\end{command}
Note that when \declare{unicode bidi} options of the package has any of the \opt{global} or \opt{local} values, then you rarely need to use any of the commands in \autoref{subsection:tslrrlt}.



\subsection{Footnote Commands}
\begin{command}{\footnote\oarg{num}\marg{text}}
Typesets right to left footnote in right to left mode and left to right footnote in left to right mode.
\end{command}

\begin{command}{\thanks\marg{text}}
Behaves like |\footnote| but only should be used inside |\title| and |\author| arguments.
\end{command}

\begin{command}{\footnotetext\oarg{num}\marg{text}}
Used in conjuction with |\footnotemark|, behaves like |\footnote|.
\end{command}

\begin{command}{\LTRfootnote\oarg{num}\marg{text}}
Always typesets left to right footnote independent on the current mode.
\end{command}

\begin{command}{\LTRthanks\marg{text}}
Behaves like |\LTRfootnote| but only should be used inside |\title| and |\author| arguments.
\end{command}

\begin{command}{\LTRfootnotetext\oarg{num}\marg{text}}
Used in conjuction with |\footnotemark|, behaves like |\LTRfootnote|.
\end{command}


\begin{command}{\RTLfootnote\oarg{num}\marg{text}}
Always typesets right to left footnote independent on the current mode.
\end{command}


\begin{command}{\RTLthanks\marg{text}}
Behaves like |\RTLfootnote| but only should be used inside |\title| and |\author| arguments.
\end{command}

\begin{command}{\RTLfootnotetext\oarg{num}\marg{text}}
Used in conjuction with |\footnotemark|, behaves like |\RTLfootnote|.
\end{command}

\subsubsection{Footnote Rule Commands}
The behaviour of the footnote rule can be controlled.

\begin{command}{\autofootnoterule}
Footnote rule is right or left aligned based on the direction of the first footnote following the rule (i.e. put in the current page).
\end{command}

\begin{command}{\rightfootnoterule}
Footnote rule is always right aligned.
\end{command}


\begin{command}{\leftfootnoterule}
Footnote rule is always left aligned.
\end{command}

\begin{command}{\textwidthfootnoterule}
Full-width footnote rule.
\end{command}


\begin{command}{\SplitFootnoteRule}
Puts a full-width rule above the split-off part of a split footnote.
\end{command}

\begin{command}{\debugfootnotedirection}
Writes the  direction of the first footnote on each page, to the terminal and the log file.
\end{command}






\section{License of the \textit{Package}}


\subsection{The GNU General Public License, Version 2}
\label{subsection:gpl}

\subsubsection{Preamble}

The licenses for most software are designed to take away your freedom to share and change it. By contrast, the \textsc{gnu} General Public License is intended to guarantee your freedom to share and change free software---to make sure the software is free for all its users. This General Public License applies to most of the Free Software Foundation's software and to any other program whose authors commit to using it. (Some other Free Software Foundation software is covered by the \textsc{gnu} Library General Public License instead.) You can apply it to your programs, too.

When we speak of free software, we are referring to freedom, not price. Our General Public Licenses are designed to make sure that you have the freedom to distribute copies of free software (and charge for this service if you wish), that you receive source code or can get it if you want it, that you can change the software or use pieces of it in new free programs; and that you know you can do these things.

To protect your rights, we need to make restrictions that forbid anyone to deny you these rights or to ask you to surrender the rights. These restrictions translate to certain responsibilities for you if you distribute copies of the software, or if you modify it.

For example, if you distribute copies of such a program, whether gratis or for a fee, you must give the recipients all the rights that you have. You must make sure that they, too, receive or can get the source code. And you must show them these terms so they know their rights.

We protect your rights with two steps: (1) copyright the software, and (2) offer you this license which gives you legal permission to copy, distribute and/or modify the software.

Also, for each author's protection and ours, we want to make certain that everyone understands that there is no warranty for this free software. If the software is modified by someone else and passed on, we want its recipients to know that what they have is not the original, so that any problems introduced by others will not reflect on the original authors' reputations.

Finally, any free program is threatened constantly by software patents. We wish to avoid the danger that redistributors of a free program will individually obtain patent licenses, in effect making the program proprietary. To prevent this, we have made it clear that any patent must be licensed for everyone's free use or not licensed at all.

The precise terms and conditions for copying, distribution and modification follow.

\subsubsection{Terms and Conditions For Copying, Distribution and Modification}

\begin{enumerate}
\addtocounter{enumi}{-1}
\item
  This License applies to any program or other work which contains a notice placed by the copyright holder saying it may be distributed under the terms of this General Public License. The ``Program'', below, refers to any such program or work, and a ``work based on the Program'' means either the Program or any derivative work under copyright law: that is to say, a work containing the Program or a portion of it, either verbatim or with modifications and/or translated into another language. (Hereinafter, translation is included without limitation in the term ``modification''.) Each licensee is addressed as ``you''.

  Activities other than copying, distribution and modification are not covered by this License; they are outside its scope. The act of running the Program is not restricted, and the output from the Program is covered only if its contents constitute a work based on the Program (independent of having been made by running the Program). Whether that is true depends on what the Program does.
\item
  You may copy and distribute verbatim copies of the Program's source code as you receive it, in any medium, provided that you conspicuously and appropriately publish on each copy an appropriate copyright notice and disclaimer of warranty; keep intact all the notices that refer to this License and to the absence of any warranty; and give any other recipients of the Program a copy of this License along with the Program.

  You may charge a fee for the physical act of transferring a copy, and you may at your option offer warranty protection in exchange for a fee.
\item
  You may modify your copy or copies of the Program or any portion of it, thus forming a work based on the Program, and copy and distribute such modifications or work under the terms of Section 1 above, provided that you also meet all of these conditions:
  \begin{enumerate}
  \item
    You must cause the modified files to carry prominent notices stating that you changed the files and the date of any change.
  \item
    You must cause any work that you distribute or publish, that in whole or in part contains or is derived from the Program or any part thereof, to be licensed as a whole at no charge to all third parties under the terms of this License.
  \item
    If the modified program normally reads commands interactively when run, you must cause it, when started running for such interactive use in the most ordinary way, to print or display an announcement including an appropriate copyright notice and a notice that there is no warranty (or else, saying that you provide a warranty) and that users may redistribute the program under these conditions, and telling the user how to view a copy of this License. (Exception: if the Program itself is interactive but does not normally print such an announcement, your work based on the Program is not required to print an announcement.)
  \end{enumerate}

  These requirements apply to the modified work as a whole. If identifiable sections of that work are not derived from the Program, and can be reasonably considered independent and separate works in themselves, then this License, and its terms, do not apply to those sections when you distribute them as separate works. But when you distribute the same sections as part of a whole which is a work based on the Program, the distribution of the whole must be on the terms of this License, whose permissions for other licensees extend to the entire whole, and thus to each and every part regardless of who wrote it.

  Thus, it is not the intent of this section to claim rights or contest your rights to work written entirely by you; rather, the intent is to exercise the right to control the distribution of derivative or collective works based on the Program.

  In addition, mere aggregation of another work not based on the Program with the Program (or with a work based on the Program) on a volume of a storage or distribution medium does not bring the other work under the scope of this License.
\item
  You may copy and distribute the Program (or a work based on it, under Section 2) in object code or executable form under the terms of Sections 1 and 2 above provided that you also do one of the following:
  \begin{enumerate}
  \item
    Accompany it with the complete corresponding machine-readable source code, which must be distributed under the terms of Sections 1 and 2 above on a medium customarily used for software interchange; or,
  \item
    Accompany it with a written offer, valid for at least three years, to give any third party, for a charge no more than your cost of physically performing source distribution, a complete machine-readable copy of the corresponding source code, to be distributed under the terms of Sections 1 and 2 above on a medium customarily used for software interchange; or,
  \item
    Accompany it with the information you received as to the offer to distribute corresponding source code. (This alternative is allowed only for noncommercial distribution and only if you received the program in object code or executable form with such an offer, in accord with Subsubsection b above.)
  \end{enumerate}

  The source code for a work means the preferred form of the work for making modifications to it. For an executable work, complete source code means all the source code for all modules it contains, plus any associated interface definition files, plus the scripts used to control compilation and installation of the executable. However, as a special exception, the source code distributed need not include anything that is normally distributed (in either source or binary form) with the major components (compiler, kernel, and so on) of the operating system on which the executable runs, unless that component itself accompanies the executable.

  If distribution of executable or object code is made by offering access to copy from a designated place, then offering equivalent access to copy the source code from the same place counts as distribution of the source code, even though third parties are not compelled to copy the source along with the object code.
\item
  You may not copy, modify, sublicense, or distribute the Program except as expressly provided under this License. Any attempt otherwise to copy, modify, sublicense or distribute the Program is void, and will automatically terminate your rights under this License. However, parties who have received copies, or rights, from you under this License will not have their licenses terminated so long as such parties remain in full compliance.
\item
  You are not required to accept this License, since you have not signed it. However, nothing else grants you permission to modify or distribute the Program or its derivative works. These actions are prohibited by law if you do not accept this License. Therefore, by modifying or distributing the Program (or any work based on the Program), you indicate your acceptance of this License to do so, and all its terms and conditions for copying, distributing or modifying the Program or works based on it.
\item
  Each time you redistribute the Program (or any work based on the Program), the recipient automatically receives a license from the original licensor to copy, distribute or modify the Program subject to these terms and conditions. You may not impose any further restrictions on the recipients' exercise of the rights granted herein. You are not responsible for enforcing compliance by third parties to this License.
\item
  If, as a consequence of a court judgment or allegation of patent infringement or for any other reason (not limited to patent issues), conditions are imposed on you (whether by court order, agreement or otherwise) that contradict the conditions of this License, they do not excuse you from the conditions of this License. If you cannot distribute so as to satisfy simultaneously your obligations under this License and any other pertinent obligations, then as a consequence you may not distribute the Program at all. For example, if a patent license would not permit royalty-free redistribution of the Program by all those who receive copies directly or indirectly through you, then the only way you could satisfy both it and this License would be to refrain entirely from distribution of the Program.

  If any portion of this section is held invalid or unenforceable under any particular circumstance, the balance of the section is intended to apply and the section as a whole is intended to apply in other circumstances.

  It is not the purpose of this section to induce you to infringe any patents or other property right claims or to contest validity of any such claims; this section has the sole purpose of protecting the integrity of the free software distribution system, which is implemented by public license practices. Many people have made generous contributions to the wide range of software distributed through that system in reliance on consistent application of that system; it is up to the author/donor to decide if he or she is willing to distribute software through any other system and a licensee cannot impose that choice.

  This section is intended to make thoroughly clear what is believed to be a consequence of the rest of this License.
\item
  If the distribution and/or use of the Program is restricted in certain countries either by patents or by copyrighted interfaces, the original copyright holder who places the Program under this License may add an explicit geographical distribution limitation excluding those countries, so that distribution is permitted only in or among countries not thus excluded. In such case, this License incorporates the limitation as if written in the body of this License.
\item
  The Free Software Foundation may publish revised and/or new versions of the General Public License from time to time. Such new versions will be similar in spirit to the present version, but may differ in detail to address new problems or concerns.

  Each version is given a distinguishing version number. If the Program specifies a version number of this License which applies to it and ``any later version'', you have the option of following the terms and conditions either of that version or of any later version published by the Free Software Foundation. If the Program does not specify a version number of this License, you may choose any version ever published by the Free Software Foundation.
\item
  If you wish to incorporate parts of the Program into other free programs whose distribution conditions are different, write to the author to ask for permission. For software which is copyrighted by the Free Software Foundation, write to the Free Software Foundation; we sometimes make exceptions for this. Our decision will be guided by the two goals of preserving the free status of all derivatives of our free software and of promoting the sharing and reuse of software generally.
\end{enumerate}

\subsubsection{No Warranty}

\begin{enumerate}
\addtocounter{enumi}{9}
\item
  Because the program is licensed free of charge, there is no warranty for the program, to the extent permitted by applicable law. Except when otherwise stated in writing the copyright holders and/or other parties provide the program ``as is'' without warranty of any kind, either expressed or implied, including, but not limited to, the implied warranties of merchantability and fitness for a particular purpose. The entire risk as to the quality and performance of the program is with you. Should the program prove defective, you assume the cost of all necessary servicing, repair or correction.
\item
  In no event unless required by applicable law or agreed to in writing will any copyright holder, or any other party who may modify and/or redistribute the program as permitted above, be liable to you for damages, including any general, special, incidental or consequential damages arising out of the use or inability to use the program (including but not limited to loss of data or data being rendered inaccurate or losses sustained by you or third parties or a failure of the program to operate with any other programs), even if such holder or other party has been advised of the possibility of such damages.
\end{enumerate}



\printindex






\end{document}